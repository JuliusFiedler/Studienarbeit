Der folgende Abschnitt orientiert sich an der Beschreibung des SINDy-Algorithmus durch Brunton et al.\cite{Brunton2016}. 

Sei $\boldsymbol{f}\left(\boldsymbol{x}(t)\right) = \diff{\boldsymbol{x}(t)}{t}$ das zu identifizierende Differentialgleichungssystem mit dem Zustandsvektor $\boldsymbol{x}(t) = [x_1(t), x_2(t), ... , x_n(t)]^T \in\mathbb{R}^n$.
Für die Anwendung von SINDy benötigt man die Messdaten aller Zustandsgrößen zu den Zeitpunkten $t_1, t_2, ..., t_m$. Zusätzlich müssen die zeitlichen Ableitungen der Zustände an den gegebenen Zeitpunkten gegeben sein, entweder durch direkte Messung oder durch numerische Approximation. Die Daten werden wie folgt angeordnet: 
\begin{align}
X &= \begin{bmatrix}
		x_1(t_1) & x_2(t_1) & \dots & x_n(t_1) \\
		x_1(t_2) & x_2(t_2) & \dots & x_n(t_2) \\
		\vdots   & \vdots   & 		& \vdots \\ 
		x_1(t_m) & x_2(t_m) & \dots & x_n(t_m)
	\end{bmatrix} \in \mathbb{R}^{m\times n},
	\\
	\dot{X} &= \begin{bmatrix} 
		\dot{x}_1(t_1) & \dot{x}_2(t_1) & \dots & \dot{x}_n(t_1) \\
		\dot{x}_1(t_2) & \dot{x}_2(t_2) & \dots & \dot{x}_n(t_2) \\
		\vdots 		   & \vdots 		& 		& \vdots \\
		\dot{x}_1(t_m) & \dot{x}_2(t_m) & \dots & \dot{x}_n(t_m)
	\end{bmatrix}  \in \mathbb{R}^{m\times n}.
\end{align}
Nun muss man die Bibliothek $\Theta$ an Funktionen konstruieren, durch welche $\boldsymbol{f}$ dargestellt werden soll. 
Die Spalten der Bibliotheksmatrix repräsentieren die gewählten Ansatzfunktionen, angewendet auf die Datenmatrix $X$
\begin{equation}
\Theta(X) = \begin{bmatrix}
		\mid & \mid & & \mid \\
		\theta_1(X) & \theta_2(X) & \dots & \theta_\ell(X) \\
		\mid & \mid & & \mid 
	\end{bmatrix}\in\mathbb{R}^{m\times L}.
\end{equation} 
Dabei steht $\ell$ für die Anzahl von verschiedenen Typen von Ansatzfunktionen und $L$ für die Gesamtzahl der Spalten von $\Theta$, welche sich aus der Wahl der Ansatzfunktionen ergibt. Wählt man beispielsweise für $\theta_1$ die Sinusfunktion und für $\theta_2$ Monome zweiten Grades so ergeben sich 
\begin{align}
\theta_1(X) &= \begin{bmatrix}
		\mid 	  & \mid     		  &          & \mid             \\
		\sin(x_1(t)) & \sin(x_2(t))   & \dots    & \sin(x_n(t)) \\
		\mid      & \mid     		  &          & \mid              
	\end{bmatrix},\\
\theta_2(X) &= \begin{bmatrix}
		\mid & \mid & & \mid & \mid & & \mid \\
		x_1(t)^2 & x_1(t)x_2(t) & \dots & x_2(t)^2 & x_2(t)x_3(t) & \dots & x_n^2(t) \\
		\mid & \mid & & \mid & \mid & & \mid
	\end{bmatrix}.
\end{align}
Gesucht sind nun die Koeffizienten $\boldsymbol{\xi}_i$ der Linearkombinationen von Bibliotheksfunktionen, sodass gilt
\begin{equation}
f_i(x) = \Theta(\boldsymbol{x}^T)\,\boldsymbol{\xi}_i.
\end{equation}
Fasst man alle $\boldsymbol{\xi}_i$ in eine Koeffizientenmatrix $\Xi\in\mathbb{R}^{L\times n}$ zusammen, so ergibt sich das zu lösende Minimierungsproblem zu 
\begin{equation}
\dot{X} \approx \Theta(X)\Xi. \label{eq:Minimierungsproblem}
\end{equation}

