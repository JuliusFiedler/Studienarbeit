Das Lösen des Gleichungssystems \eqref{eq:Minimierungsproblem} ist die Kernaufgabe des Algorithmus. Das Gleichungssystem besitzt $m\cdot n$ Gleichungen ($m\text{ } \hat{=}$ Anzahl Messungen) und $L\cdot n$ Unbekannte ($L\text{ } \hat{=}$ Anzahl Ansatzfunktionen). Unter der Annahme, dass die Anzahl der Messungen ausreichend groß ist, handelt es sich um ein überbestimmtes Gleichungssystem. 
Brunton et. al. \cite{Brunton2016}  schlägt einen sequentiellen Algorithmus (\textit{Sequentially Thresholded Least Squares algorithm} STLSQ) vor. Eine Abwandlung davon wurde im Verlauf dieser Arbeit in \textit{Python} entwickelt und wird im Folgenden vorgestellt.
