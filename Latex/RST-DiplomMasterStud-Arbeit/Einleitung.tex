\section{Motivation}
Für viele regelungstechnische Anwendungen ist die Kenntnis von Systemmodellen erforderlich. Unter Nutzung von Methoden des maschinellen Lernens können aus Messdaten Modelle erzeugt werden, die das Verhalten des Systems widerspiegeln können. Oft handelt es sich um Black-Box-Modelle, die zwar das Eingangs-Ausgangs-Verhalten des Systems wiedergeben, jedoch keinen Aufschluss zu den zugrunde liegenden Differentialgleichungen liefern können. Diese mathematischen Zusammenhänge können jedoch essenziell im Verständnis und im Umgang mit komplexen Systemen sein. In dieser Arbeit wird die Methode \textit{Sparse Identification of Nonlinear Dynamics} (SINDy) untersucht, die verspricht die Systemdifferentialgleichungen aus Messdaten des Systems zu rekonstruieren. 

\section{Präzisierung der Aufgabenstellung}
Aus den existierenden Methoden zur Systemidentifikation sollen vielversprechende Kandidaten ausgewählt werden. Diese sollen während der Literaturrecherche auf Eignung und Implementierbarkeit geprüft werden. 
Anschließend sollen wichtige Ergebnisse aus der Literatur reproduziert und weitere Beispielsysteme steigender Komplexität übertragen werden. 
Dabei soll der Einfluss der zur Verfügung stehenden Freiheitsgrade herausgearbeitet werden.
