Eine weit verbreitete Strategie zur Lösung überbestimmter Gleichungssysteme der Form
\begin{equation}
\dot{X} = \Theta\Xi
\end{equation} ist die Methode der kleinsten Quadrate. Dabei wird die Moore-Penrose-Inverse von $\Theta$ berechnet
\begin{equation}
\Theta^{+} = (\Theta^T\Theta)^{-1}\Theta^T.
\end{equation}
Die Moore-Penrose-Lösung
\begin{equation}
\Xi = \Theta^+\dot{X}
\end{equation}
minimiert dabei den Fehler
\begin{equation}
\epsilon = \norm{\Theta(X)\Xi-\dot{X}}_2. \label{eq:Fehler_MKQ}
\end{equation} 
Der Algorithmus beginnt mit dieser Näherungslösung für die Koeffizientenmatrix $\Xi$.
Um die Forderung nach einer dünnbesetzten Matrix umzusetzen, werden anschließend alle Koeffizienten, deren Betrag unter einem festgelegten Grenzwert liegt, zu Null gesetzt.
\begin{equation}
\Xi_\text{dünn, ij} = \begin{cases} 0 & |\Xi_\text{ij}| < \lambda\\
\Xi_\text{ij} & \text{sonst} 
\end{cases} ,\quad 1 \leq i \leq L, \quad 1\leq j \leq n \label{eq:make_sparse}
\end{equation}
Jeder nicht negative Koeffizient $\Xi_\text{dünn, ij}$ repräsentiert eine im Differentialgleichungssystem vorkommende Ansatzfunktion. Allerdings sind die Koeffizienten $\Xi_\text{dünn}$ fehlerhaft, da sie auf Basis der nicht vorkommenden Funktionen berechnet wurden. Es bietet sich an, die Koeffizientenmatrix $\Xi$ erneut zu berechnen, unter Nutzung des Wissens über welche Ansatzfunktionen in der Differentialgleichung (DGL) vorkommen.

Für jede Zeile der DGL wählt man diejenigen Spalten der Bibliotheksmatrix aus, die für Funktionen stehen, die nach \ref{eq:make_sparse} in dieser Zeile vorkommen. Dann kann man die Koeffizientenmatrix spaltenweise wie folgt berechnen (der Index a steht für Auswahl, die Indices der $\theta$ sind willkürlich gewählt):
\begin{subequations}
\begin{equation}
\dot{X}_i = 	\left(\begin{array}{c} 
      					 \mid \\
      					 \dot{x}_i(t)\\ 
      					 \mid 
    				\end{array}\right) \in \mathbb{R}^{m}
\end{equation}
\begin{equation}
\Theta_\text{a}(X) = \begin{bmatrix}
		\mid & \mid &  \mid &\\
		\theta_2(X) & \theta_5(X) & \theta_7(X) &\dots\\
		\mid & \mid &  \mid &
	\end{bmatrix}\in\mathbb{R}^{m\times l}.
\end{equation}
\begin{equation}
\hat{\Xi}_i = \xi_i \in\mathbb{R}^{l} = \Theta_\text{a}^+\dot{X}_i \label{eq:xi_select}
\end{equation}
\end{subequations}
Die Spalten $\hat{\Xi}_i$ können zu einer Matrix $\hat{\Xi}$ zusammengesetzt werden. Anschließend muss diese auf  $\Xi_\text{neu}\in\mathbb{R}^{L\times n}$ vergrößert und an den entsprechenden Stellen mit Nullen aufgefüllt werden: 
\begin{equation}
\Xi_\text{neu, ij} = \begin{cases}
						0 & |\Xi_\text{ij}| < \lambda\\
						\hat{\Xi}_\text{kj}, 1\leq k \leq l  & \text{sonst}

\end{cases}. \label{eq:reshape} % ist das mathematisch sauber?
\end{equation}
Allerdings besteht die Möglichkeit, dass $\Xi_\text{neu}$ durch die erneute Berechnung Einträge unterhalb des Grenzwertes besitzt.
Daher werden die Schritte \ref{eq:make_sparse}, \ref{eq:xi_select} und \ref{eq:reshape} solange wiederholt, bis in \ref{eq:make_sparse} $\Xi_\text{dünn} = \Xi$ gilt, also bis die Koeffizientenmatrix nicht mehr verändert wird. Es werden somit iterativ immer mehr Ansatzfunktionen ausgeschlossen. Damit gilt am Ende
\begin{subequations}
\begin{equation}
\dot{X} \approx \Theta(X)\Xi
\end{equation}
und somit
\begin{equation}
\dot{x} = f(x) \approx \Xi^T\left(\Theta(x^T)\right)^T,
\end{equation}
wobei $\Xi$ dünn besetzt ist.

\end{subequations}













