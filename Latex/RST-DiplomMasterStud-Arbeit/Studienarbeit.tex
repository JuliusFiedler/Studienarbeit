\documentclass[arbeit=studie,oneside,BCOR=12mm]{ArbeitRST}
% Die Option BCOR legt den Rand für die Bindekorrektur links fest
% (verschiebt das ganze Dokument nach rechts auf dem Papier, damit
% Platz zum Binden ist

% bib-Datei mit den Literaturangaben
% ==================================
\addbibresource{Literatur-Arbeit.bib}

% Paket zum Erzeugen von Blindtext 
% ================================
% (ist nur für dieses Beispieldokument sinnvoll)
\usepackage{blindtext}



% Zwei Parameter zum Verändern des Layouts
% ========================================
% \parindent -> Legt fest, mit welcher Einrückung jeder neue
%               Absatz beginnen soll
% \parskip -> Legt fest, wieviel vertikaler Abstand zwischen zwei
%             Absätzen liegen soll
%
% Tipp: Entweder parindent auf Null und parskip auf einen Wert
% ungleich Null (z.B. 2ex) oder umgekehrt. Beide Werte ungleich
% Null macht satztechnisch keinen Sinn. 1ex = Breite des 
% Buchstabens x
\setlength{\parindent}{0ex}
\setlength{\parskip}{2ex}


% Einige Einstellungen für das hyperref-Paket
% =========================================== 
% Hiermit können Links, Gleichungsnummern etc. farbig dargestellt
% werden was die Navigation im elektronischen Dokument vereinfacht. An
% dieser Stelle können Sie die Farbgebung anpassen. Druckversion bitte
% ohne farbige Links erstellen, siehe Option unten!
\hypersetup{
    unicode=false,          % non-Latin characters in Acrobat’s bookmarks
    pdftoolbar=true,        % show Acrobat’s toolbar?
    pdfmenubar=true,        % show Acrobat’s menu?
    pdffitwindow=false,     % window fit to page when opened
    pdfstartview={FitH},    % fits the width of the page to the window
    pdftitle={RST Vorlage}, % title
    pdfauthor={Author},     % author
    pdfsubject={Subject},   % subject of the document
    pdfcreator={Creator},   % creator of the document
    pdfproducer={Producer}, % producer of the document
    pdfkeywords={keyword1} {key2} {key3}, % list of keywords
    pdfnewwindow=true,      % links in new window
    colorlinks=true,        % false: boxed links; true: colored links
    linkcolor=blue,         % color of internal links (change box color with linkbordercolor)
    citecolor=green,        % color of links to bibliography
    filecolor=magenta,      % color of file links
    urlcolor=cyan           % color of external links
}

% Entfernt die farbigen Markierungen - bitte Druckversion mit dieser Option kompilieren
%\hypersetup{hidelinks}

   

% =================================================================
\begin{document}

% Titelseite
% ==========

% Name des Verfassers
\author{Julius Fiedler}

% Geburtsort
\geburtsort{Dresden}

% Geburtsdatum
\geburtsdatum{13. Oktober 1996}

% Titel der Arbeit
\title{Über den Einfluss hochfrequenter mechanischer Oszillationen auf das Schaltverhalten supraleitender PID-Regler auf Quantenbasis}

% Untertitel
\subtitle{Eine Fallstudie unter besonderer Berücksichtigung stochastischer Einflüsse}

% Angabe der Betreuer
\betreuer{Betreuer 1}
\betreuer{Betreuer 2}

% Datum der Einreichung
\date{2. Februar 2222}


% Zunächst für das Vorgeplänkel römische Seitenzahlen und einfacher Seitenstil
% ============================================================================
\pagenumbering{Roman}
\pagestyle{plain}


% Titelseite erstellen
\maketitle


% Selbstständigkeitserklärung
% ===========================

% Ort der Selbstständigkeitserklärung (Standard: Dresden)
\selbstort{Pirna}

% Datum der Selbstständigkeitserklärung (Standard: aktuelles Systemdatum)
\selbstdatum{1. Januar 2016}

% Selbstständigkeitserklärung erstellen
\selbststaendigkeitserklaerung


% Kurzfassung / Abstract
% ======================
\kurzfassung{An dieser Stelle fügen Sie bitte eine deutsche Kurzfassung ein.}{Please insert the English abstract here.}


% Inhaltsverzeichnis
% ==================
\tableofcontents

% Ggf. Symbolverzeichnis
% ======================
\chapter*{Verzeichnis der Formelzeichen \markboth{VERZEICHNIS DER FORMELZEICHEN}{}} \label{ch:Symbolverzeichnis}
\addcontentsline{toc}{chapter}{Verzeichnis der Formelzeichen}

\begin{table}[htbp]
\centering
\begin{tabular}{llp{9 cm}}
$C^r(n,k)$ & Kombination mit Wiederholung\\
$\varepsilon$ & Fehler der Moore-Penrose-Lösung\\
$\varepsilon_\text{r}$& relativer Fehler der identifizieren Koeffizienten\\
$\boldsymbol{f}$ & Systemdynamik\\
$F$ & Kraft, die auf den Wagen wirkt\\
$g$ & Erdbeschleunigung\\
$m$ & Anzahl der Messungen\\
$m_1$ & Masse des Wagens\\
$m_2$ & Masse am Ende des Pendels\\
$n$ & Anzahl der Zustände\\
$L$ & Anzahl der Ansatzfunktionen in $\Theta$\\
$\boldsymbol{p}$ & Parametervektor\\
$P$ & nominale Koeffizientenmatrix\\
$\varphi$ & Auslenkung des Pendels aus der unteren Ruhelage\\
$Q$ & identifizierte Koeffizientenmatrix\\ 
$R$ & Hilfsmatrix zur Berechnung von $\varepsilon_\text{r}$\\
$s$ & Länge des Pendels\\
$t$ & Zeit \\
$\theta$ & Gruppe von Ansatzfunktionen\\
$\Theta$ & $\Theta(X)$\\
$\Theta(X)$ & Bibliotheksmatrix aus Ansatzfunktionen, ausgewertet für alle von $X$\\
$\Theta^+$ & Moore-Penrose-Inverse von $\Theta$\\
$\Theta_\text{v}$ & verkleinerte Bibliotheksmatrix\\
$U$ & Unbekannte Funktion\\
$x$ & Position des Wagens\\
$\boldsymbol{x}(t)\in\mathbb{R}^n$ & Zustandsvektor\\
$X$ & Matrix der zeitlichen Verläufe der Zustandskomponenten\\
$\dot{X}$ & Matrix der zeitlichen Verläufe der Ableitungen der Zustandskomponenten\\
$\xi$ & Spalte von $\Xi$\\
$\Xi$ & Koeffizientenmatrix\\
$\Xi^\text{d}$ & dünn besetzte Koeffizientenmatrix\\
$\hat{\Xi}$ & verkleinerte Koeffizientenmatrix unter Nutzung von $\Theta_\text{v}$\\
$\Xi^n$ & Koeffizientenmatrix in den Dimensionen von $\Xi$ mit den Einträgen von $\hat{\Xi}$\\

\end{tabular}
\end{table}

% Ggf. Abbildungsverzeichnis
% ==========================
\listoffigures


% Ggf. Tabellenverzeichnis
% ========================
\listoftables


% ========================
% Beginn Inhalt der Arbeit
% ========================

% Inhalt kann entweder in separate LaTeX-Dateien ausgelagert werden
% (hier: inhalt.tex) und dann mittels \input{} geladen werden...
% Darauf achten, dass pagenumbering und pagestyle richtig gesetzt werden
% (siehe Einträge in inhalt.tex)


%\input{inhalt}


% ...oder man schreibt direkt in dieser Datei (weniger übersichtlich)
\chapter{Notizen}

\section{UDE}

\begin{figure}[ht]
\begin{subfigure}[c]{0.5\textwidth}
\centering
\includegraphics[width=1\textwidth]{images/arctan_1x_order_100}

\end{subfigure}
\begin{subfigure}[c]{0.5\textwidth}
\centering
\includegraphics[width=1\textwidth]{images/arctan_10x_order_100}

\end{subfigure}
\begin{subfigure}[c]{0.5\textwidth}
\centering
\includegraphics[width=1\textwidth]{images/tanh_1x_order_100}

\end{subfigure}
\begin{subfigure}[c]{0.5\textwidth}
\centering
\includegraphics[width=1\textwidth]{images/tanh_10x_order_100}

\end{subfigure}
\end{figure}


\subsection{8.7.20} 
sindy bei wp kann koeff nicht mehr schätzen grund unklar\\
überlegung NN in julia und sindy in python machen, export erledigt\\
trotzdem keine identifikation in python möglich, grund: zu wenig daten (datenfeld mit 31 daten Autflösung viel zu gering)\\
bei erhöhung der auflösung wird trainingszeit enorm groß
NN approximiert den verlauf der ableitungen	\\
sindy mal mit anderen anfangswerten trainieren
\subsection{9.7.20}
\subsubsection{reprodutierbarkeit udesindy}
\subsubsection{ude reibung}
\subsubsection{Pysindy reibung}

% ==================================
% Literaturverzeichnis
% ==================================

% Ein Literatureintrag, der nicht referenziert wird, aber im Verzeichnis erscheinen soll
\nocite{Mik57de}

% Literaturverzeichnis ausgeben
\printbibliography

\end{document}


%%% Local Variables:
%%% mode: latex
%%% TeX-master: t
%%% End:
