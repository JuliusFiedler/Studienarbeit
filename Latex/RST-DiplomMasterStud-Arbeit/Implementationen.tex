Neben dem in Abschnitt \ref{sec:rMKQ} vorgestellten vereinfachten Algorithmus werden zwei weitere Implementationen der SINDy-Methode untersucht. De Silva et al. \cite{desilva2020} haben SINDy in \textit{Python} (PySINDy) implementiert, während Rackauckas et al.\cite{Rackauckas2020} die SINDy-Implementation der DataDrivenDiffEq-Bibliothek in \textit{Julia} nutzen. Die drei Algorithmen sind sich im Aufbau sehr ähnlich und folgen dem in Kapitel \ref{cha:sindy} beschriebenen Ablauf. Jedoch unterscheiden sich die Algorithmen in der Lösungsstrategie des überbestimmten Gleichungssystems \eqref{eq:Axeqb}. PySINDy verwendet das Cholesky-Verfahren \cite{Krishnamoorthy2011}, während die Julia-Implementation die QR-Faktorisierung \cite{Martinsson2015} anwendet. 