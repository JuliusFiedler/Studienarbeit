
Sparse Identification of Nonlinear Dynamics (SINDy) ist eine Methode, um aus den Messdaten eines Systems auf dessen Systemdifferentialgleichungen zu schließen. Die zugrundeliegende Annahme ist, dass die zu identifizierende Funktion in einem geeigneten Raum an Basisfunktionen \textit{dünnbesetzt} ist. Betrachtet man beispielsweise die Funktion
\begin{equation}
\diff{\boldsymbol{x}}{t} = \boldsymbol{f}(\boldsymbol{x}) = \begin{bmatrix} f_1(\boldsymbol{x})\\ f_2(\boldsymbol{x})\\ \end{bmatrix}
=\begin{bmatrix} 1+2x_1+x_1x_2^2 \\ x_1^3-3x_2\\ \end{bmatrix},
\end{equation}
so ist leicht zu erkennen, dass $\boldsymbol{f}$ in Bezug auf die Basis von Polynomen aus zwei Variablen (z.B. $f_1(\boldsymbol{x})=\sum_{i=0}^{\infty}\sum_{j=0}^{\infty}a_{ij}x_1^ix_2^j$) dünnbesetzt ist. Nur eine sehr geringe Zahl der Koeffizienten $a_{ij}$ ist ungleich null. 
Die dem Algorithmus zur Verfügung gestellten Basisfunktionen werden zu einer Bibliothek zusammengefasst. Hieraus werden mittels Regression diejenigen Ansatzfunktionen ausgewählt, deren Linearkombination die Funktion $\boldsymbol{f}$ am besten repräsentiert. Die Auswahl der Bibliotheksfunktionen spielt dabei eine entscheidende Rolle für den Erfolg der Methode. Daher ist es günstig, wenn man bereits im Voraus gewisse Kenntnisse zu den zu identifizierenden Funktionsklassen besitzt, um die Bibliothek geeignet auslegen zu können. Wie der Methodenname suggeriert, können mit SINDy auch nichtlineare Funktionen identifiziert werden. 