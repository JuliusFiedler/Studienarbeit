Für den Vergleich der Implementationen werden drei verschiedene Differentialgleichungssysteme betrachtet, deren Parameter es zu identifizieren gilt.

Das \textit{Lotka-Volterra}-System (im Folgenden auch kurz: Volterra-System) ist definiert durch
\begin{equation}
\begin{aligned}
\dot{x} &= \alpha x +\beta xy\\
\dot{y} &= \gamma y + \delta xy \label{eq:volterra}
\end{aligned}
\end{equation}
mit den Nominalparametern
\begin{equation}
\boldsymbol{p}_\text{V}=	\begin{pmatrix*}[c]
				\alpha&\beta&\gamma&\delta
				\end{pmatrix*} = 
				\begin{pmatrix*}[c]
					1.3& -0.9&  -1.8 & 0.8
				\end{pmatrix*}
\end{equation}
und den Anfangswerten
\begin{equation}
\begin{pmatrix*}[c]
				x_0 & y_0
				\end{pmatrix*} = 
				\begin{pmatrix*}[c]
					0.442 & 4.62
				\end{pmatrix*}.
\end{equation}

Das \textit{Lorenz}-System ist definiert durch
\begin{equation}
\begin{aligned}
\dot{x} &= \alpha(y-x)\\
\dot{y} &= x(\beta - z) -y\\
\dot{z} &= xy-\gamma z
\end{aligned}
\end{equation}
mit den Nominalparametern
\begin{equation}
\boldsymbol{p}_\text{L}=	\begin{pmatrix*}[c]
				\alpha&\beta&\gamma
				\end{pmatrix*} = 
				\begin{pmatrix*}[c]
					10& 28& \frac{8}{3}
				\end{pmatrix*}
\end{equation}
und den Anfangswerten
\begin{equation}
\begin{pmatrix*}[c]
				x_0 & y_0 & z_0
				\end{pmatrix*} = 
				\begin{pmatrix*}[c]
				-8& 8& 27
				\end{pmatrix*}.
\end{equation}

Das \textit{Rössler}-System ist definiert durch
\begin{equation}
\begin{aligned}
\dot{x} &= -y-z\\
\dot{y} &= x+\alpha y\\
\dot{z} &= \beta + (x-c)z
\end{aligned}
\end{equation}
mit den Nominalparametern
\begin{equation}
\boldsymbol{p}_\text{R}=	\begin{pmatrix*}[c]
				\alpha&\beta&\gamma
				\end{pmatrix*} = 
				\begin{pmatrix*}[c]
					0.2& 0.1& 5.3
				\end{pmatrix*}
\end{equation}
und den Anfangswerten
\begin{equation}
\begin{pmatrix*}[c]
				x_0 & y_0 & z_0
				\end{pmatrix*} = 
				\begin{pmatrix*}[c]
				1& -1& -1
				\end{pmatrix*}.
\end{equation}
Ausschlaggebend für die Güte des jeweiligen Algorithmus ist an erster Stelle die Genauigkeit des Ergebnisses, an zweiter Stelle die Rechenzeit. Um die Genauigkeit des Algorithmus vergleichen zu können, wird ein Fehler $\varepsilon_\text{r}$  definiert. Dieser ist an das quadratische Mittel des relativen Fehlers angelehnt. $P\in\mathbb{R}^{L\times n}$ bezeichnet die nominale und $Q\in\mathbb{R}^{L\times n}$ die identifizierte Koeffizientenmatrix eines Systems. Dann sei
\begin{equation}
R \in\mathbb{R}^{L\times n} , \quad R_{ij}= \begin{cases}
|\frac{P_{ij}-Q_{ij}}{P_{ij}}|, & P_{ij} \neq 0\\
|Q_{ij}|, & P_{ij} = 0 \wedge |Q_{ij}| < 1\\
1, & P_{ij} = 0 \wedge |Q_{ij}| \geq 1\\
0, & \text{sonst}
\end{cases}, \quad  1 \leq i \leq L, \quad 1\leq j \leq n
\end{equation}
\begin{equation}
\varepsilon_\text{r} = \frac{1}{\sqrt{k}}\norm{R}_2,
\end{equation} 
wobei $k$ die Anzahl der von Null verschiedenen Elemente von $P$ ist. Für richtig identifizierte Ansatzfunktionen wird somit der relative Fehler zwischen Nominalparameter und identifiziertem Parameter berechnet, während bei falsch identifizierten Ansatzfunktionen der Fehler, bis hin zu einer Grenze, vom identifizierten Parameter abhängt. Dadurch werden kleine Koeffizienten vor falschen Funktionen weniger stark bestraft, da diese einen geringeren Einfluss auf das Ergebnis haben.

Die Systeme werden durch einen Differentialgleichungslöser simuliert und die zeitlichen Verläufe der Zustandskomponenten numerisch ermittelt. Die Ableitungsverläufe werden sowohl exakt vorgegeben, als auch über die Zentraldifferenz angenähert. Die exakten Ableitungsverläufe dienen hier nur der Überprüfung, ob die Methode theoretisch funktioniert und sind im Praxisfall aufgrund von Messfehlern oft nicht gegeben. Das Hauptaugenmerk wird auf der Identifikation unter Nutzung der Zentraldifferenz liegen, da diese den Praxisfall besser abbildet.






