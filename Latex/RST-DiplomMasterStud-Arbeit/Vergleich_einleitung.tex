Für den Vergleich der Implementationen werden drei verschiedene Differentialgleichungssystem betrachtet, deren Parameter es zu identifizieren gilt.

Das Lotka-Volterra-System
\begin{equation}
\begin{aligned}
\dot{x} &= \alpha x +\beta xy\\
\dot{y} &= \gamma y + \delta xy 
\end{aligned}
\end{equation}
mit den Nominalparametern
\begin{equation}
p_\text{LV}=\left(\begin{array}{c}
		\alpha\\\beta\\\gamma\\\delta
\end{array}\right) 
= \left(\begin{array}{c}
		1.3\\ -0.9\\ 0.8\\ -1.8
\end{array}\right).
\end{equation}

Das Lorenz-System
\begin{equation}
\begin{aligned}
\dot{x} &= \alpha(y-x)\\
\dot{y} &= x(\beta - z) -y\\
\dot{z} &= xy-\gamma z
\end{aligned}
\end{equation}
mit den Nominalparametern
\begin{equation}
p_\text{L}=\left(\begin{array}{c}
		\alpha\\\beta\\\gamma
\end{array}\right) 
= \left(\begin{array}{c}
		10\\ 28\\ \frac{8}{3}
\end{array}\right).
\end{equation}

Das Rössler-System
\begin{equation}
\begin{aligned}
\dot{x} &= -y-z\\
\dot{y} &= x+\alpha y\\
\dot{z} &= \beta + (x-c)z
\end{aligned}
\end{equation}
mit den Nominalparametern
\begin{equation}
p_\text{R}=\left(\begin{array}{c}
		\alpha\\\beta\\\gamma
\end{array}\right) 
= \left(\begin{array}{c}
		0.2\\ 0.1\\ 5.3
\end{array}\right).
\end{equation}

Ausschlaggebend für die Güte des jeweiligen Algorithmus ist an erster Stelle die Genauigkeit des Ergebnisses, an zweiter Stelle die Rechenzeit. Um die Genauigkeit des Algorithmus vergleichen zu können, wird ein Fehler $\epsilon_r$  definiert. $P\in\mathbb{R}^{L\times n}$ bezeichnet die nominalen und $Q\in\mathbb{R}^{L\times n}$ die identifizierten Parameter eines Systems.
\begin{equation}
R \in\mathbb{R}^{L\times n} , \quad R_{ij}= \begin{cases}
\frac{P_{ij}-Q_{ij}}{P_{ij}} & P_{ij} \neq 0\\
\frac{P_{ij}-Q_{ij}}{Q_{ij}} & P_{ij} = 0 \wedge Q_{ij} \neq 0\\
0 & \text{sonst}
\end{cases}, \quad  1 \leq i \leq L, \quad 1\leq j \leq n
\end{equation}
\begin{equation}
\epsilon_r = \frac{1}{\sqrt{Ln}}\norm{R}_2
\end{equation} 
Für jedes System werden die zeitlichen Verläufe der Zustände durch einen Differentialgleichungslöser numerisch %??? 
ermittelt. Die Ableitungsverläufe werden sowohl exakt vorgegeben, als auch über die Zentraldifferenz angenähert. Dabei wird der Einfluss folgender Parameter auf das Identifikationsergebnis untersucht:
\begin{itemize}
\item die Simulationszeit $t$ des DGL-Lösers
\item die Schrittweite $dt$ des DGL-Lösers
\item die Verwendung mehrerer Datenreihen auf einmal 
\item die Gestaltung der Bibliothek aus Ansatzfunktionen
\end{itemize}




