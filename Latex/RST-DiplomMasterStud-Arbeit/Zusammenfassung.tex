\section{Zusammenfassung}
In dieser Arbeit wurde der SINDy-Algorithmus als Methode zur Gewinnung von Systemdifferentialgleichungen aus Messdaten untersucht. Es wurde der Einfluss verschiedener Parameter auf das Identifikationsergebnis vorgestellt und Richtlinien zu deren Einstellung abgeleitet. SINDy ist besonders für die Identifikation parameterlinearer Systeme geeignet, darüber hinaus muss man viel Vorwissen beisteuern, um brauchbare Ergebnisse zu erzielen. Allgemein ist das benötigte Vorwissen über das zu untersuchende System ein entscheidender Nachteil der Methode. Ist dieses jedoch vorhanden, so kann SINDy sehr genaue Ergebnisse liefern, was z.B. die Schätzung von Systemparametern angeht.

\section{Ausblick}
Potential für weitere Untersuchungen bietet der SINDy-Algorithmus bei der Identifikation teilweise bekannter Systeme. Oft sind die physikalischen Zusammenhänge eines Systems theoretisch bekannt. SINDy kann dann dazu genutzt werden, nur parasitäre Einflüsse wie Reibungen zu modellieren. 
Eine weitere vorstellbare Anwendung könnte die Echtzeitschätzung von Parametern sein. Die Messdaten der Zustandskomponenten werden zur Laufzeit, z.B. durch einen Beobachter, gesammelt und mittels SINDy werden die Systemparameter aktualisiert. 